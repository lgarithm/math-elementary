\section{Series of Real Numbers}
\subsection{Sequence of Real Numbers}
A finite sequence of $n$ real numbers is a map $a : \{0, \dots, n - 1\} \to \mathbb R$, usually written as $a_0, \dots, a_{n - 1}$, or $\{a_i\}_{i=0}^{n-1}$.

An infinite sequence of real numbers is a map $a : \mathbb N \to \mathbb R$,
usually written as $a_0, \dots, a_n, \dots$, or $\{a_i\}_{i=0}^\infty$.

We will also use $a_n$ to denote a sequence of real numbers, either finite or
infinite. For a sequence of real numbers $a_n$, its partial sum is another
sequence of real numbers $b_n$ where $b_n = \sum_{i=0}^n a_i$.

\begin{defi}[limit of infinite sequence]
Let $a, a_n \in \mathbb R$, $a$ is the limit of $a_n$ if for any $\epsilon > 0$, there exists $N \in \mathbb N$ such that $\abs{a_n - a} < \epsilon$ when $n > N$.
\end{defi}

The next proposition is a result of $\mathbb R$ being a Hausdorff topological space.
\begin{pro}
Every infinite sequence of real numbers has at most one limit.
\end{pro}

\begin{defi}[convergence]
A sequence of real number is convergent if it has a limit.
\end{defi}

\subsection{Cauchy Sequence}

\begin{defi}
A sequence $a_n$ is a Cauchy sequence if and only if for any $\epsilon > 0$,
exist $N$ such that for all $m, n > N$, $\abs{a_m - a_n} < \epsilon$.
\end{defi}

The next proposition is a result of $\mathbb R$ being a complete metric space.
\begin{pro}
A sequence of real numbers is convergent if and only if it's a Cauchy sequence.
\end{pro}

The definition of Cauchy sequence also applies to sequence of rational numbers,
but a Cauchy sequence of rational number may not converges to a rational number.
For two sequence $a_n$, $b_n$, define relation $a_n \sim b_n$ if and only if
$a_n - b_n$ converges to $0$.
It's trivial to see that $\sim$ is an equivalent relation.
$\mathbb R$ can be defined as the equivalent classes of Cauchy sequences in $\mathbb Q$.

\subsection{Upper and Lower Limit}

\begin{defi}
Let $\{a_n\}$ be a sequence, define $A_n = \max\{a_k \mid k \geq n\}$,
the limit of $\{A_n\}$, if exist, is called the upper limit of $\{a_n\}$.
\end{defi}

\begin{defi}
Let $\{a_n\}$ be a sequence, define $B_n = \min\{a_k \mid k \geq n\}$,
the limit of $\{A_n\}$, if exist, is called the lower limit of $\{a_n\}$.
\end{defi}


\subsection{Series}
A series is the sum of an infinite sequence: $$\sum_{n=0}^\infty a_n$$
the partial sum of a series is just the partial sum of $a$:
$$S_n = \sum_{k=0}^n a_k$$
A series is convergent if its partitial sum is convergent.
\begin{pro}
If $\sum_{n=0}^\infty a_n$ convergence, then $\lim_{n \to \infty} a_n = 0$.
\end{pro}

\begin{pro}
$\sum_{n=0}^\infty x_n$ is convergence if and only if for any $\epsilon > 0$,
exist $N$, such that for all $n > m > N$, $\abs{a_{m+1} + \cdots + a_n} < \epsilon$.
\end{pro}

\begin{pro}
The harmonic series $$\sum_{n=1}^\infty {1 \over n}$$ is divergent.
\end{pro}

\begin{pro}
For $a \in \mathbb R$, the geometric series $$\sum_{n=0}^\infty a^n$$ is
convergent when $\abs a < 1$, and divergent when $\abs a \geq 1$.
More specifically, it converges to $1 \over 1 - a$ when $\abs a < 1$.
\end{pro}

\begin{pro}
The series $$\sum_{n=0}^\infty {1 \over n!}$$ is convergent.
\end{pro}
