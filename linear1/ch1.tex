\section{Matrix}
An $m$-by-$n$ matrix is simply $mn$ numbers arranged in $m$ rows and $n$ colomns:
$$\left(\begin{matrix}
a_{11} & \cdots & a_{1n} \cr
\vdots & \ddots & \vdots \cr
a_{m1} & \cdots & a_{mn}
\end{matrix}\right)$$

\subsection{Matrix Operation}
For two matrix of the same size, if addition is defined on their elements,
we can define their sum as termwise sum
$$\left(\begin{matrix}
a_{11} & \cdots & a_{1n} \cr
\vdots & \ddots & \vdots \cr
a_{m1} & \cdots & a_{mn}
\end{matrix}\right) +
\left(\begin{matrix}
b_{11} & \cdots & b_{1n} \cr
\vdots & \ddots & \vdots \cr
b_{m1} & \cdots & b_{mn}
\end{matrix}\right)
=\left(\begin{matrix}
a_{11} + b_{11} & \cdots & a_{1n} + b_{1n} \cr
\vdots & \ddots & \vdots \cr
a_{m1} + b_{m1} & \cdots & a_{mn} + b_{mn}
\end{matrix}\right)$$
For a $k$-by-$m$ matrix $A = (a_{ij})$ and an $m$-by-$n$ matrix $B = (b_{ij})$,
if both addition and multiplication are deifned on their elements\footnote{
Usually we interested in a matrix whose elements are from a commutative ring.},
we can define their product $AB$ as a $k$-by-$n$ matrix $C = (c_{ij})$, where
$$c_{ij} = \sum_{k=1}^m a_{ik}b_{kj}$$
One can check by definition that the associative law $$A(BC) = (AB)C$$
and distribution law
\begin{align*}
A(B + C) &= AB + AC \\
(A + B)C &= AB + BC
\end{align*}
hold if their sizes are proper to be added and multiplied.
But the commutative law $AB = BA$ doesn't hold in general.

If $A = (a_{ij})$ is an $m$-by-$n$ matrix,
we define the transpose of $A$ as a $n$-b-$m$ matrix,
$A^\prime = (a^\prime_{ij})$, where $a^\prime_{ij} = a_{ji}$.

\begin{rem}
All matrixes over commutative ring $R$ can be considered as a category,
where the objects are positive integers, and the morphisms from $m$ to $n$ are
the $m$-by-$n$ matrixes. Note that $\hom(m, n)$ forms an abelian group, the
transpose defines a group isomorphism between $\hom(m, n)$ and $\hom(n, m)$.
And $\hom(n, n)$ forms an $R$-algebra. When $R = K$ is a field, $\hom(n, n)$
is a $K$-linear space of dimension $n^2$.
\end{rem}

\subsection{Row-Column Transformation}
For given $m, n$, let $\mat_{mn}(R)$ denote the set of $m$-by-$n$ matrixes over $R$,
or simply $\mat_{mn}$. Define the action of $S_m$ on $\mat_{mn}$ by permuting
the rows, i.e.
\begin{equation*}
\sigma :
\left(\begin{matrix}
a_{11} & \cdots & a_{1n} \cr
\vdots & \ddots & \vdots \cr
a_{m1} & \cdots & a_{mn}
\end{matrix}\right)
\to
\left(\begin{matrix}
a_{\sigma(1) 1} & \cdots & a_{\sigma(1) n} \cr
\vdots & \ddots & \vdots \cr
a_{\sigma(m) 1} & \cdots & a_{\sigma(m) n}
\end{matrix}\right)
\end{equation*}

A permutation matirx is a matirx such that $a_{ij} = \delta(i, \sigma(j))$ for
some permutation $\sigma$.
\begin{pro}
The action of $\sigma$ on matrix $A$ is equivalent to multiply $A$ by the
permutation matirx $\delta(i, \sigma^{-1}(j))$ on the left.
\end{pro}

The column transformation has similar results to row transformation.

\subsection{Triangular Matrix}
A matrix $A = (a_{ij})$ is upper trangular if $a_{ij} = 0$ when $i > j$,
is lower trangular if $a_{ij} = 0$ when $i < j$.
A is called a diagonal matrix if $a_{ij} = 0$ when $i \neq j$.

\begin{pro}
For any Euclid ring $R$, and $A \in \mat_{mn}(R)$, there exists a square matrix
$T \in \mat_{mm}$ such that $TA$ is an upper triangular matrix.
\end{pro}

\begin{cly}
For any matrix $A$ over an Euclid ring, there exists $P$, $Q$ such
that $PAQ$ is a diagonal matrix.
\end{cly}

\subsection{Square Matrix}
